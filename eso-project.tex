\documentclass[12pt,journal,compsoc]{IEEEtran}

\usepackage[portuguese]{babel}
\usepackage[utf8]{inputenc}

\begin{document}

\title{Carros elétricos para um mundo melhor}
\author{
  Daniel~M.~Assis, 
  Lindomar~G.~Ferreira, 
  Rodrigo~Magalhães
  \IEEEcompsocitemizethanks{
    \IEEEcompsocthanksitem D. M. Assis, L. Ferreira e R. Magalhães são mestrandos do Instituto de Pesquisas Tecnológicas do Estado de São Paulo
  }
}

\IEEEcompsoctitleabstractindextext{%
%\begin{abstract}
%Abstract aqui
%\end{abstract}
\begin{IEEEkeywords}
  Engenharia de Software, SWEBOK
\end{IEEEkeywords}}

\maketitle

\IEEEdisplaynotcompsoctitleabstractindextext
\IEEEpeerreviewmaketitle


\section{Introdução ao Projeto}

\IEEEPARstart{N}{os} dias atuais, poucas coisas são mais importantes para a Humanidade quanto a questão da ecologia. Na proporção em que crescemos enquanto espécie, a projeção média de crescimento da população mundial é de 9 bilhões de pessoas até 2050. \cite{wwf_living_2013} Esse crescimento populacional influi diretamente nas condições de vida no planeta.

Desta forma, no que concerne à área de atuação da MotorBRAS, a redução no controle de emissões de gás carbônico apresenta-se, mais do que nunca, como um diferencial competitivo atraente. Países mais desenvolvidos estão prontos e dispostos a investir em tecnologias para redução do consumo de CO2, de forma que o risco no investimento em tecnologias ambientais apresenta-se como baixo, uma vez que a demanda por novas tecnologias neste campo é cada vez mais atrativa. As empresas que forem pioneiras no investimento de tecnologias ambientais que influenciem a vida das pessoas terão uma ótima vantagem competitiva em relação aos concorrentes. 

Este projeto visa agregar valor à intenção da MoToRBRAS de entrar no mercado de carros elétricos como serviço até 2020. A proposta é a de agregar tecnologia ao produto, de forma a cativar o consumidor em relação ao serviço e, assim, mitigar o impacto do Homem no meio ambiente. Consequências diretas são a geração de uma imagem positiva no mercado, bem como a conquista de uma posição de vanguarda e liderança no mercado. 

Países desenvolvidos são favoráveis à ideia de aluguel de carros elétricos em suas grandes cidades, como forma de locomoção barata e rápida das pessoas para seu local de trabalho. Carros elétricos, sendo pequenos e de baixo consumo, são altamente atrativos como um meio-termo entre os carros particulares e o transporte público. 

O que este projeto propõe é tornar esse serviço ainda mais atrativo, através da customização de cada carro, pelo uso de tecnologia de ponta. O diferencial é permitir que o carro seja automaticamente customizado para o indivíduo. 


\section{Condições e Características Importantes}

De acordo com o projeto previso pelo grupo, os seguintes características serão agregadas aos carros elétricos:

- Reconhecimento de íris: uma maneira eficiente de identificar o indivíduo para então iniciar o processo de customização de suas preferências;
- Integração com redes sociais: para que seja possível ao indivíduo ter acesso a sua agenda, contatos, calendários e preferências musicais instantaneamente; 
- Customizações físicas internas: preferências como temperatura do carro (ajuste do ar condicionado) e posição do banco serão carregados de um servidor central e aplicados diretamente ao veículo quando da identificação do indivíduo; da mesma forma, alterações no veículo são salvas no servidor central como preferências do indivíduo, para posterior aplicação numa nova viagem;

Além das características acima citadas, outros aspectos devem ser considerados para garantir o correto funcionamento de todo o processo, tais como:

- Segurança: um aspecto importante que deve ser considerado em todo o projeto, tanto pelo aspecto de segurança da identidade do indivíduo quanto pela segurança na execução das tecnologias físicas de customização e carregamento automático, bem como pela segurança nas transações de aluguel do veículo;
- Controle cadastral: o gerenciamento da customização das preferências e do pagamento pelo serviço são aspectos que também devem ser endereçados, de forma que um sistema para controle cadastral dos clientes faz-se necessário.

Além dos sistemas internos ao veículo, são necessários softwares externos para compor a infraestrutura de prestação de serviços. Esses softwares são:

- Sistema para aluguel de veículos pelos clientes - componente importante da infraestrutura de prestação do serviço, este software permitirá que os motoristas interessados efetuem agendamento e posterior pagamento pelo serviço de aluguel de veículo através da internet;
- Sistema para gestão da frota de veículos - o processo de gestão da frota de veículos requer um software de apoio. Atividades de manutenção, abastecimento, disponibilização e recepção de veículos nos pátios, entre outras, serão suportadas por esse software. Este software representará o elo entre o processo de aluguel de veículos e as atividades administrativas e operacionais ligadas à gestão da frota.
- Servidor de armazenamento de perfis - software executado em um servidor centralizado que armazenará as preferências de cada usuário, bem como efetuará, de forma segura, a ponte entre o veículo e os dados disponíveis nas redes sociais do motorista.

Algumas características do produto e do contexto de negócios influenciarão a formatação do projeto. A listagem de características e respectivas consequências para o projeto são:

- Alto nível de inovação do produto - não existem serviços similares no mercado brasileiro. Estamos tomando como base a experiência de serviços semelhantes em outros países, mas, devido a fatores culturais, existe a possibilidade da experiência não se repetir com sucesso em nosso mercado. Tal situação possivelmente levará a uma proposta de adaptação pelo departamento de marketing, o que, por sua vez, exigirá a capacidade de absorção de novos requisitos por parte da TI. Dessa forma, optamos pela adoção de métodos ágeis, que são mais adequados a esse tipo de ambiente de negócio.
- Heterogeneidade dos softwares a serem desenvolvidos - diferentes tipos de software serão desenvolvidos - desde sistemas desktop administrativos, passando por softwares para servidores e chegando a software embarcado para execução dentro do veículo. Esse contexto torna interessante a ideia da criação de pequenas equipes, especializadas no domínio específico da aplicação que estão desenvolvendo, dedicadas à atuação apenas no software que atende suas respectivas especialidades.


\section{Disciplinas do SWEBOK-3}

As disciplinas do SWEBOK serão aplicadas pelos times de desenvolvimento conforme segue: 

\subsection{Requisitos}

Os requisitos de produto serão expressos por meio de uma especificação composta de histórias [1], por ordem de prioridade, que são definições funcionais de alto nível. O analista de requisitos será responsável por prover informações detalhadas referentes às histórias aos demais membros do time, no momento de estimativa da iteração e durante toda a iteração. Desta forma, cabe a este profissional ter o entendimento dos detalhes das histórias. Será entregue, a cada iteração, uma documentação que detalha os aspectos importantes de cada história realizada. [verificar questao de prender o texto com o que vem abaixo por gancho no fim do parágrafo]

Engenheiro de software com perfil mais técnico também precisará ser envolvido no entendimento dos requisitos juntamente com o analista de requisitos, pois a visão técnica que considera os requisitos não-funcionais deve ser analisada ao mesmo tempo em que os requisitos surgem.

O tester também deve estar no levantamento de requisitos, para prover uma visão mais funcional que permita que a definição da especificação seja definida de uma forma funcional e testável.

Os analistas de requisito precisarão ter livre acesso aos stakeholders (pessoal de marketing e parte jurídica, que atuam como reguladores), que vão fornecer e auxiliar na validação dos requisitos. Esse acesso deverá ser feito através de entrevistas, onde podem-se fazer uso de protótipos pela técnica de DDD [2] (mais do que o rigor do UML) e exemplificação de cenários como formas de obter clareza nos detalhes das histórias. Os modelos conceituais estão a cargo dos entrevistadores, e não definem característica formal. Apenas é esperado que sejam definidos cenários para cada história [3], de forma a guiar o time durante a construção e permitir geração de validação automatizada dos requisitos. [4]

Considerando a característica dinâmica da equipe de marketing, assume-se que os requisitos são voláteis; a abordagem para mudanças é a de tentar incorporar o máximo possível se a história permanecer a mesma, mas caso trate-se de outra história, incorporar apenas na próxima iteração, como forma de não impactar o time.

Uma arquitetura será definida com base no entendimento das primeiras histórias da especificação (e com base em conhecimento da visão geral do projeto e características de infra-estrutura da empresa), e evoluída durante todo o projeto [5], apoiando-se em técnicas como refatoração e automatização de testes de regressão.

Os requisitos serão validados através de testes de aceitação, de duas formas: manual (na reunião de demonstração da iteração, quando os stakeholders comparam as histórias combinadas em relação ao funcionamento do sistema), e automatizada: utilizando a técnica de BDD [6], cada cenário de cada história será vinculado a métodos de teste durante a construção, de forma que cada execução da suite de testes irá gerar um relatório que define se as histórias estão funcionando de acordo com o esperado.


\subsection{Design}

No início do projeto (antes da primeira iteração), considerando a visão geral e o entendimento dos requisitos de maior prioridade, o design de software arquitetural (top-level) será definido por um engenheiro de software com perfil sênior em desenvolvimento. Após isto, a cada iteração, o trabalho de design detalhado será incorporado no trabalho dos desenvolvedores da equipe, de forma que o design não será mais dissociado da construção, e sim será parte desta [7]. O engenheiro sênior em desenvolvimento pode definir design arquitetural para futuras iterações, dependendo da necessidade, e trabalhar juntamente com o time na definição do design detalhado, durante a iteração. O design arquitetural deverá ser documentado em alto nível, considerando aspectos como estilo arquitetural sendo utilizado, forma de tratamento de eventos, forma de lidar com concorrência, distribuição de componentes, persistência de dados.

Já a responsabilidade de design detalhado cabe a todos os desenvolvedores do time, e não será exigida documentação formal para tal, pois assume-se o conceito de que o "código-fonte é o design" [8]. Contudo, será necessário validar a qualidade deste design, o que será feito através da elaboração de relatório de inspeção de código a cada iteração, que deve considerar as questões do documento de design arquitetural, além de técnicas de design, como: abstração, acoplamento, coesão, decomposição, modularização, encapsulamento, separação de responsabilidades, design patterns. 

O relatório será realizado pelo engenheiro de software da própria equipe que agregar o papel de sênior, e deve ser entregue como um artefato da iteração (relatório de qualidade de design). A responsabilidade pela veracidade do documento é de toda a equipe.

Além deste relatório, a qualidade do design também será garantida através de:
- práticas diárias de DDD: para permitir uma modelagem do design alinhada com as regras de negócio sendo utilizadas. O DDD faz uso de um diagrama simplificado de classes para modelar o entendimento;
- práticas diárias de pair programming e TDD [9]: que orientam o desenvolvedor no sentido de escrever um código claro e seguindo o paradigma de orientação a objetos;
- práticas automatizadas: análise estática de código, para garantir que boas práticas de design estão sendo adotadas no código sendo escrito [10]. Esta prática não deve ser avaliada apenas no final da iteração, mas diariamente, por todo o time.

Em relação às descrições estruturais, espera-se que o documento de arquitetura defina diagramas de deployment, e é desejável que os diagramas de classes simplificados que surgem da prática de DDD também sejam documentados (a ser definido pelo time).

Para este projeto, a estratégia de design é baseada em design orientado a objeto, por ser um modelo comumente difundido pelos membros do time e por ser adequar ao projeto proposto.


\subsection{Construção}

Este projeto tratará design e construção de forma muito próxima. Os engenheiros de software com perfil de programadores terão poder de definir o design detalhado bem como a codificação, sempre orientados por um programador sênior e pelo design arquitetural. O programador poderá, portanto, aplicar os fundamentos de minimização de complexidade, antecipação de mudanças e construção para verificação tanto em design quanto em construção, sem a necessidade de haver distinções.

É esperado que o programador reduza a complexidade em tudo o que for fazer, utilizando os melhores algoritmos e padrões. Também é esperado que o código seja escrito de forma a permitir futuras modificações sem grande impacto (e para isso, é exigido que a prática de TDD seja adotada, e o conceito de refatoração seja entendido [11]). Outras práticas que serão exigidas do programador são a escrita de testes unitários e de integração [12]. A verificação da adoção destas práticas será realizada juntamente com a validação de qualidade do design, descrita no tópico anterior. 

Também será exigido que o programador escreva testes de aceitação baseados em BDD, contra cada história da especificação que implementa, para garantir que seu código seja verificável. Estes testes serão executados diariamente, gerando relatórios que indicam se o código é válido em relação ao esperado. Será exigido que o programador siga os padrões de desenvolvimento de código da empresa, e a ferramenta de análise estática de código irá indicar se os padrões estão sendo ignorados, diariamente, durante a iteração. Códigos não-conformes devem ser corrigidos o mais rápido possível, no prazo máximo de duração da iteração.

Não será exigido uso formal de UML; contudo, o time pode optar por utilizar UML (bem como qualquer outra forma visual de design) para facilitar a comunicação e/ou o entendimento de um dado objeto ou conjunto de objetos que precisa ser criado ou alterado. 

O modelo de construção que será adotado neste projeto é o Scrum. Este modelo iterativo trata a construção de forma concorrente a outras atividades de desenvolvimento de software, como os requisitos, o design e os testes. Todas estas atividades interagem juntamente nesta fase de construção; desta forma, o planejamento da construção se dá iterativamente, envolvendo todas as atividades: desde a fase de análise (com a definição dos requisitos da especificação como histórias por ordem de prioridade, e o entendimento profundo destas através da elaboração de cenários), passando por  uma estimativa para a iteração (onde time discute e define quantas histórias da especificação podem ser realizadas, considerando a prioridade) e pela construção propriamente dita (onde os programadores geram código considerando design, os analistas documentam as histórias e os testers validam as histórias contra o sistema conforme vão sendo finalizadas), e finalizando na entrega (onde os stakeholder validam, numa reunião de demonstração, o sistema rodando, a documentação e demais artefatos que tenham sido previamente combinados a serem entregues). Para medida das iterações, são utilizadas as métricas já descritas nos tópicos anteriores, bem como o próprio aval dos stakeholders.


\subsection{Testes}

A verificação dinâmica dos testes será aplicada pelo engenheiro de software com perfil de tester durante cada iteração, assim que uma data história for dada como pronta pelos engenheiros de software com perfil de programadores. Para este fim, os programadores devem liberar cada história pronta num servidor de testes.

A verificação, num primeiro momento, deve contemplar o conjunto de cenários da história (teste de conformidade, ou teste funcional) e a análise dos relatórios automatizados gerados pelo framework de BDD.

Como já dito, o tester deve participar na fase de levantamento de requisitos, para auxiliar o analista na definição das histórias e cenários específicos; neste ponto, em sua área de conhecimento, garante critérios de seleção e adequação dos testes, bem como o objetivo e a efetividade dos testes. O ganho adicional com qualidade vem do fato de considerar que os programadores escrevem a aplicação partindo da mesma ótica, o que auxilia na redução de falhas de comunicação e de entendimento.

Deseja-se que o tester tenha perfil de programador, no sentido de poder auxiliar na revisão de testes unitários e de integração que os desenvolvedores tenham criado. Considerando a característica de pair programming da equipe, pode até mesmo auxiliar o programador nesta atividade. Contudo, esta atividade não é obrigatória, sendo esperado que o foco do tester mais voltado ao aspecto funcional do software. 

Outro aspecto em que o tester deve atuar é em testes de sistema, verificando se diversos requisitos não-funcionais estão de acordo com o esperado (com o apoio do documento de design arquitetural e dos programadores), além de analisar aspectos como performance, stress, configuração, usabilidade. Espera-se que faça uso de  ferramentas para automatizar tais testes, e que gere um relatório no final de cada iteração, referente a estes testes específicos.

Os testes de regressão funcionais devem ser automatizados, cobrindo os aspectos funcionais de maior importância (a serem definidos pelo próprio tester). Os aspectos não-funcionais não são esperados em relação à regressão, pois sua característica não sugere regressão, mas sim diagnóstico pontual.

O tester tem liberdade para decidir realizar testes estruturais, testes exploratórios e demais técnicas de testes que julgar válidas na iteração. Estes testes também precisam ser documentados em relatório, com justificativa e resultados. Os testes precisam ser planejados no início da iteração, e tal definição precisa também constar no relatório a ser entregue no final da iteração.


\subsection{Manutenção}

A equipe que dará manutenção é a mesma que desenvolveu o produto. Desta forma, qualquer nova requisição de modificação (evolução ou correção) deve ser definida na especificação, seja alterando ou adicionando novos itens. Tal item, então, precisa ser definido em forma de história e com prioridade, para ingressar numa próxima iteração, seguindo o mesmo fluxo de um novo requisito. Desta forma, incorporando correções no fluxo normal de desenvolvimento (e não após todo o projeto ter terminado), procura-se eliminar o problema onde 80 por cento do esforço de manutenção é gasto em ações não corretivas. Neste caso, a manutenção acontece enquanto o projeto ainda existe, e a prioridade fica a cargo dos stakeholders. 

A evolução do software, portanto, será agregada durante o projeto de desenvolvimento, cabendo apenas que haja um gerenciamento dos requisitos em relação à prioridade. Como a equipe é a mesma, não há transição de conhecimento. As manutenções corretivas, adaptativas e perfectivas irão entrar no fluxo normal de desenvolvimento. Já espera-se que a manutenção preventiva seja um trabalho implícito, resultante das práticas de qualidade e métricas nas fases de análise, design e construção. No caso especial da construção, espera-se aplicação de refatoração e code review.

A definição de testes de regressão de outras fases também irá ajudar na manutenção, definindo aspectos que não funcionam mais conforme esperado. Os testes de aceitação automatizados com BDD também garantem a mesma regressão, mas num nível mais alto: que as funcionalidades continuam válidas. 

Espera-se que toda a qualidade definida até o momento (com testes automatizados, métricas e inspeção de design e análise) permitam que uma análise de impacto gere sempre um baixo custo, e que haja uma alta capacidade de manutenibilidade. 

Após a finalização do projeto, uma equipe menor será designada para manter o mesmo seguindo os mesmos princípios definidos até então, e com boa parte do mesmo pessoal, de forma a facilitar a transmissão de conhecimento. Esta futura equipe, poderá fazer uso dos documentos de especificação e design de arquitetura e das documentações de testes para entender o sistema de um aspecto funcional, e deverá estudar o código para entender um design mais detalhado. O esforço em qualidade é pago aqui, pois o código deve estar escrito de forma que seja fácil de entender e evoluir, de forma que a prática de engenharia reversa deve ser desconsiderada. 

As práticas de inspeção de qualidade não devem deixar de existir, sob pena de degradar a qualidade do código, de forma que a futura equipe de manutenção precisará contar com um engenheiro de software com perfil sênior para estar responsável por este aspecto (utilizando as ferramentas necessárias para tal fim).


\subsection{Gerenciamento de Configuração}

Haverá uma gerência de configuração responsável pelo armazenamento de todos os artefatos gerados nas fases de requisitos, design, construção, testes e manutenção, bem como ferramentas necessárias a todas as áreas. O planejamento desta gerência deve considerar apenas elementos de software, descartando hardware e firmware por não serem aspectos cujo controle faz-se necessário. Não caberá à gerência de configuração definir regras sobre os artefatos gerados (com exceção do código-fonte gerado), de forma que os times podem versionar livremente; contudo, precisa prover mecanismos para auditar a informação (entende-se que estes mecanismos devem ser manuais, pela diversidade de artefatos que um dado time pode produzir). 

Em relação ao código-fonte gerado, a gerência de configuração precisará garantir sua qualidade através da análise estática de código com uso de ferramenta automatizada, de forma a garantir que o código está seguindo as boas práticas determinadas pela empresa. Relatórios desta medida de qualidade de código devem estar disponível para consulta a qualquer momento. O desenvolvedor deve possuir acesso a estas mesmas práticas, de forma a poder criar um código-fonte de acordo com as práticas exigidas pela gerência de configuração. 

Outra característica que precisa ser considerada é a permissão de acesso. Apenas usuários de um dado projeto podem ter permissão de acesso ao repositório de informações do projeto, a não ser que a gerência diga o contrário. Assim sendo, é preciso haver uma forma automática de conceder acessos, tanto individuais quanto a grupos, de forma que um privilégio individual de um elemento num grupo possa sobrescrever seu privilégio enquanto membro do grupo.

Esta gerência também será responsável pela integração contínua: deverá prover e manter uma ferramenta de geração de builds automatizados, a ser executada diariamente ("integrando" o código, agregando a execução de testes unitários e de integração gerados pelos programadores ao processo de build, gerando relatórios de métrica de cobertura de testes, e notificando todo o time no caso de builds e testes quebrados), bem como prover uma opção para geração de releases e deploy automático em servidor de testes (base da entregacontínua). Com isto, espera-se agilidade na entrega do sistema para validação pelos stakeholders. 

Portanto, é responsabilidade da gerência de configuração gerar os releases/baselines. As baselines devem ser geradas a cada fim de iteração e os releases devem estar disponíveis para ser gerados sempre que a equipe de construção finalizar uma iteração, de forma que o ciclo de execução de trabalho deve estar alinhado com o desta equipe. Se o código-fonte não estiver conforme o validado pela análise estática de código e de acordo com as métricas de cobertura de testes geradas pela integração contínua, o release não poderá ser entregue. Cabe ao gerente de configuração a responsabilidade de alertar o time, de forma a evitar atrasos num entendimento tardio de que o release não poderá ser entregue por falta de qualidade.


\subsection{Gerenciamento de Engenharia de Software}

referencia 
  http://www.amazon.com/Agile-Project-Management-Creating-Innovative/dp/0321658396

  laboratorio (projeto):

  - medida
  - meta
    - nao tem problema de custo por ter um patrocinador rico
      - prazo - quando eh estrategico
      - funcionalidade - o que associado ao quando
  - risco 
      - ameacas do projeto
      - pontos de controle (teoria dos fatores criticos de sucesso)
          - fazer uma pesquisa dos fatores criticos de sucesso pode ser uma boa 
              - teoria parece se prender a algumas para gerenciar sempre
      - como gerenciar o ricso - medida
  - medida

ver zotero - para gerenciamento de referencias



\subsection{Gerenciamento de Processo de Software}

Justificar agile: 
http://audacium.com/wp-content/uploads/2012/08/Gartner-Agile-Maturity-Model.pdf
http://blogs.gartner.com/michael\_blechar/2010/06/27/agile-methodologies-hype-or-reality/

[ver referencia do boehm e turner, using risk to balance agile and plan-driven methods (computer june 2003)]

Para que possa o projeto possa ser viabilizado, será necessária a montagem de times de desenvolvimento. Seguindo a metodologia ágil de desenvolvimento SCRUM, os times serão dispostos em células multidisciplinares, conforme segue:

- 1 celula de infraestrutura de comunicações: composta por 1 Product Owner, 2 analistas de requisitos, 3 programadores e 1 tester
- 1 celula de customização lógica: composta por 1 Product Owner, 2 analistas de requisitos, 3 programadores e 1 tester
- 1 celula de customização física: composta por 1 Product Owner, 2 analistas de requisitos, 3 programadores e 1 tester
- 1 celula de gestão financeira e aluguel de carro: composta por 1 Product Owner, 2 analistas de requisitos, 3 programadores, 1 DBA e 1 tester 

Os analistas de requisitos precisam ter um forte entendimento do negócio e das tendências da empresa, bem como acesso livre à pessoas da parte operacional, da forma que um grupo de stakeholders precisará ser definido. Caberá aos Analistas, juntamente com os Product Owners de cada célula, a função de interagir com os Stakeholders e definir limites e restrições necessárias para o andamento do projeto.

[colocar esse trecho na parte relativa a processo]

Esta abordagem de passagem de conhecimento por comunicação em detrimento da passagem de conhecimento por documento previamente escrito pauta-se no exemplo da filosofia do sistema de produção da Toyota, que sugere a redução de custo pela eliminação de desperdício [1] (neste caso, o tempo). O trabalho colaborativo do time baseado no entendimento prova-se mais produtivo do que a construção de software baseado em documento previamente escrito. [citar refs] [incorporar historia do inicio do scrum por schwartz tambem]

Esta abordagem de aumentar produtividade pela visão dos requisitos pautados em confiança (mais do que por contratos burocráticos) torna-se mais viável pela característica de que todas as partes envolvidas no projeto são internas à MoToRBras.  



\section{Considerações Gerais}

O projeto aqui proposto caminha lado-a-lado com a tendência atual de preocupação com o peso da vida humana na Terra. É assumido que os governos irão cada vez mais investir em tecnologia para resolver tais problemas, ao invés de considerar que haja uma mudança drástica de consciência no ser humano de forma geral. Este projeto assume que nos próximos anos esta tendência irá se acentuar, de forma a surgir uma oportunidade no mercado que favorecerá as empresas de vanguarda. Nosso projeto caminha nesta direção. 



\section{Conclusão}

Este artigo procurou demonstrar como as diferentes disciplinas de engenharia de software expostas no SWEBOK podem ser aplicadas na resolução de um problema hipotético. A forma como as disciplinas do SWEBOK são abordadas tem um enfoque particular em práticas ágeis de desenvolvimento de software, de forma a demonstrar que tais práticas podem aplicadas dentro das orientações propostas pelo SWEBOK.

\bibliographystyle{IEEEtran}
\bibliography{eso-project}

\end{document}



